% [2]: Seite 11
\chapter{Deklaration}

Folgender Abschnitt beschreibt die Vorkenntnisse des Kandidaten und dessen Vorbereitung.

\section{Vorkenntnisse}

Der Lernende hat seit Praktikumsbeginn (2. August 2021) mit Ruby on Rails gearbeitet. Auch wurden die Tests in dieser Zeit mit RSpec und Capybara geschrieben und die Code-Qualität mit Rubocop überprüft.
Ebenfalls seit Praktikumsstart wurde Git und Git Flow zusammen mit Github eingesetzt. Code-Reviews gehören ebenfalls zur täglichen Arbeit (sowohl Code Reviews durchführen, wie auch entgegennehmen).
Heroku und SemaphoreCI werden ebenfalls seit Praktikumsbeginn eingesetzt.

\section{Vorarbeiten}

Zur Vorbereitung der IPA wird ein vollständiges App-Setup nach dem Renuo App-Setup-Guide gemacht.

\textbf{Dafür wird folgendes gemacht:}
\begin{itemize}
    \item Generieren einer vanilla Ruby on Rails Applikation mittels \mintinline{bash}{rails new}
    \item Einrichten von GitHub repository und allen Branches nach GitFlow
    \item Installieren von grundlegenden Abhängigkeiten. Dazu gehören z.B. die Testumgebung \emph{rspec},
          das CSS-Framework \emph{bootstrap} oder \emph{simple\_form} für ein vereinfachtes generieren von Formularen.
    \item Einrichten der CI/CD-Pipelines auf SemaphoreCI inkl. Deployment auf Heroku
    \item Den E-Mail Service Sparkpost einbinden
    \item Einen AWS S3 Storage Bucket einrichten, um die File-Uploads speichern zu können.
\end{itemize}

\section{Neue Lerninhalte}

Die Aufgabenstellung beinhaltet grundsätzlich keine neuen Lerninhalte. Allenfalls hat der Kandidat noch nie mit formatierten Text-Inhalten und verschiedenen States gearbeitet. Dies ist aber mithilfe von etwas Recherche gut lösbar.

\section{Arbeiten in den letzten 6 Monaten}

Im Herbst 2021 wurde hauptsächlich mit Ruby on Rails gearbeitet. Das grösste Projekt war eine Plattform (core-values) um die zentralen Werte einer Firma zu ermitteln.
Dazu wurde Rails und Stimulus verwendet. In den letzten Monaten wurde in einem Kundenprojekt mit Java (Spring) und Angular (Typescript) gearbeitet, daneben auch mit Ruby on Rails an unserem internen Projekt “Gifcoins”.

\section{Eingesetzte Tools}

Ein modernes MacBook Pro mit der neusten macOS Version wird sowohl für die Entwicklung als auch für die Dokumentation eingesetzt.

\subsection{Dokumentation und Projektmanagement}

\begin{itemize}
    \item Um die Arbeitszeiten des Kandidaten einzutragen, wird redmine mit dem Tracky Plugin verwendet.
    \item Der IPA-Bericht wird in LaTeX verfasst. Dazu verwendet der Kandidat Visual Studio Code mit diversen Erweiterungen.
          Der Bericht baut auf der Vorlage \cite{Buhler_ipa-template_2022} auf.
    \item Sowohl UMLet als auch das textbasierte Tool mermaid werden für UML-Diagramme verwendet.
\end{itemize}

\subsection{Entwicklung}

\begin{itemize}
    \item Die (Haupt-) Entwicklungsumgebung, die zum Einsatz kommt, ist JetBrains RubyMine.
    \item Fork wird als grafischer Git-Client eingesetzt.
    \item iTerm \& Alacritty als Terminal
    \item Der Kandidat verwendet sowohl Firefox als auch einen Chromium-basierten Browser während der Entwicklung, um eine Kompatibilität über alle Browser sicherzustellen.
\end{itemize}
