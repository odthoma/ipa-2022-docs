\chapter{Entscheiden} \label{ch:decide}

Dieses Kapitel soll

Dazu werden Nutzwertanalysen verwendet, um so möglichst objektive Entscheidungen treffen zu können. Die Analysen vergleichen jeweils
zwei Varianten von Softwarebibliotheken miteinander, die für die jeweiligen Problemstellungen infrage kommen würden. Die Kriterien werden
mittels eines Punktesystems von 0 bis 5 Punkten bewertet, welche im Anschluss zusammengezählt werden.
Jedes dieser Kriterien hat eine andere Gewichtung, bezogen auf deren Wichtigkeit in dieser PA.

Dabei spielen die bereits vorhandenen \emph{Kentnisse} über die einzelnen eine sehr grosse Rolle, da das Projekt in einer sehr kurzen Zeit realisiert werden soll.

\section{State management}
Um die in die in der  Abbildung \ref{fig:state-diagram} beschriebenen Zustände zu verwalten,
ergibt es durchaus Sinn, das Konzept einer \emph{State-Machine} einzuführen. In der Ruby on Rails Welt
gibt es mehrere Softwarebibliotheken, die dieses Konzept umsetzten.

\subsubsection{Variante 1: \enquote{AASM}}

\subsubsection{Variante 2: \enquote{Stateful Enum}}

\subsubsection{Nutzwertanalyse}

Aus der folgenden Nutzwertanalyse

\begin{tabular}{|l|l|l|l|}
    \hline
    \rowcolor{PrimaryColor!50} Kriterium & Gewichtung & Variante 1 & Variante 2 \\
    \hline
    Bekanntheit                          & 5 \%       & 4          & 2          \\
    \hline
    Kenntnisse                           & 50 \%      & 0          & 0          \\
    \hline
    Dokumentationen/Anleitungen          & 15 \%      & 5          & 5          \\
    \hline
    Komplexität                          & 10 \%      & 2          & 3          \\
    \hline
    Umfang/Features                      & 20 \%      & 4          & 2          \\
    \hline
    \rowcolor{PrimaryColor!50}           &            &            &            \\
    \hline
    SUMME inkl. Gewichtung               &            &            &            \\
    \hline
\end{tabular}

\section{Authorisierung}


\subsubsection{Variante 1: \enquote{CanCanCan}}

\subsubsection{Variante 2: \enquote{Pundit}}

\subsubsection{Nutzwertanalyse}

Aus der folgenden Nutzwertanalyse

\begin{tabular}{|l|l|l|l|}
    \hline
    \rowcolor{PrimaryColor!50} Kriterium              & Gewichtung & Variante 1 & Variante 2 \\
    \hline
    Bekanntheit                                       &            &            &            \\
    \hline
    Kenntnisse                                        &            &            &            \\
    \hline
    Dokumentationen/Anleitungen                       &            &            &            \\
    \hline
    Komplexität                                       &            &            &            \\
    \hline
    Umfang/Features                                   &            &            &            \\
    \hline
    \rowcolor{PrimaryColor!50} SUMME inkl. Gewichtung &            &            &            \\
    \hline
    Nutzwert                                          &            &            &            \\
    \hline
\end{tabular}
