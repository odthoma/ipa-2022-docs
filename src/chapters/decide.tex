\chapter{Entscheiden} \label{ch:decide}

Die meisten Entscheidungen bezogen auf das Gesamtsystem und die zu verwendenden Softwarebibliotheken werden bereits durch das
\emph{Renuo Application-Setup-Guide} abgenommen. Dennoch gibt es verschiedene Varianten von Softwarebibliotheken und Konzepten,
zwischen denen in diesem Kapitel möglichst objektiv entschieden werden soll. Dazu werden simple Nutzwertanalysen eingesetzt.

Diese Nutzwertanalysen vergleichen die vorliegenden Varianten basierend auf 6 Kriterien, welche anhand eines Punktesystems mit jeweils 0 bis 5 Punkten bewertet werden.
Jedes dieser Kriterien hat eine andere Gewichtung, bezogen auf deren Wichtigkeit in der Umsetzung dieser PA.
Dabei spielen die bestehenden Kenntnisse des Kandidaten über die einzelnen Varianten eine sehr grosse Rolle,
da das Projekt in einer sehr kurzen Zeit umgesetzt werden soll und werden daher besonders stark gewichtet.

\section{State management}
Um das Verwalten der in dem Diagramm \ref{fig:state-diagram} beschriebenen Zustände eises Assessments zu vereinfachen,
wird das Konzept einer \emph{State-Machine} eingeführt. Eine solche \enquote{Zustandsmaschine} liest eine Reihe von Eingaben
und ändert den Zustand des jeweiligen Assessments bei jeder Eingabe entsprechend. Ausserdem ermöglicht dieses Konzept einen nachvollziehbaren
Fluss und garantiert einen sauberen, lesbaren Code. In der Ruby-Welt gibt es Unmengen an Softwarebibliotheken, die ein solches Konzept implementieren.
Für diesen Entscheid wurden zwei der beliebtesten gems basierend auf deren \enquote{GitHub Stars} ausgewählt.

\subsubsection{Variante 1: \enquote{stateful\_enum}}

Diese Variante ermöglicht eine perfekte Integration in das Framework, indem sie auf der \emph{ActiveRecord::Enum} \gls{dsl} aufbaut und somit eine einfache Syntax bietet.
Es werden \emph{events} definiert, welche Übergänge zwischen Zuständen beschreiben und einen optionalen \emph{before} Callback haben.

\begin{figure}[H]
\begin{codebox}
\begin{minted}{ruby}
class Bug < ApplicationRecord
  enum status: { unassigned: 0, assigned: 1, resolved: 2 } do
    event :assign do
      transition :unassigned => :assigned
    end

    event :resolve do
      before do
        self.resolved_at = Time.zone.now
      end
      transition [:unassigned, :assigned] => :resolved
    end
  end
end
\end{minted}
\end{codebox}
\caption{\label{fig:stateful-enum-example}Beispiel einer Zustandsmaschine mit dem \enquote{stateful\_enum} gem \cite{stateful_enum_github}}
\end{figure}

\subsubsection{Variante 2: \enquote{state\_machines}}

Mit fast 100 Millionen Downloads \cite{statstate_machines-activemodel_rubygems} ist diese Variante bestimmt eine der Bekanntesten Lösungen für Zustandsmaschinen in Ruby.
Das gem ist sehr umfangreich und bietet viele Funktionalitäten in einer eigenen DSL. Es gibt verschiedenste Forks dieses gems,
die auch perfekt mit dem Ruby on Rails Framework integrieren. So gibt es beispielsweise Model-Validierungen über eine \emph{ActiveModel::Validations} Integration,
verschiedenste Arten von Callbacks, parallele Events u.v.m.

\begin{figure}[H]
\begin{codebox}
\begin{minted}{ruby}
class Vehicle
  state_machine :initial => :parked do
    before_transition :parked => any - :parked, :do => :put_on_seatbelt
    after_transition any => :parked do |vehicle, transition|
      vehicle.seatbelt = 'off'
    end
    around_transition :benchmark

    event :ignite do
      transition :parked => :idling
    end

    state :first_gear, :second_gear do
      validates_presence_of :seatbelt_on
    end
  end

  ...
end
\end{minted}
\end{codebox}
\caption{\label{fig:state-machine-example}Beispiel einer Zustandsmaschine mit dem \enquote{state\_machines} gem \cite{state_machines-activemodel_github}}
\end{figure}

\subsubsection{Nutzwertanalyse}

Die zwei Varianten unterscheiden sich sehr stark durch den Umfang an Features die sie bieten,
jedoch wird für diese PA nur eine simple Form einer Zustandsmaschine benötigt. Durch die einfach gehaltene Syntax
und die gute Integration in \emph{ActiveRecord::Enum} ist die 1. Variante der klare Gewinner der folgenden Nutzwertanalyse.

\begin{table}[H]
  \rowcolors{2}{gray!10}{white}
  \begin{tabular}{|l|l|l|l|}
    \hline
    \rowcolor{PrimaryColor!30} \textbf{Kriterium} & \textbf{Gewichtung} & \textbf{Variante 1} & \textbf{Variante 2} \\
    \hline
    Kenntnisse                                    & 40 \%               & 1                   & 0                   \\
    \hline
    Dokumentationen/Anleitungen                   & 20 \%               & 5                   & 5                   \\
    \hline
    Komplexität                                   & 10 \%               & 5                   & 2                   \\
    \hline
    Umfang/Features                               & 15 \%               & 2                   & 5                   \\
    \hline
    Integration in das Framework                  & 15 \%               & 5                   & 4                   \\
    \hline
    \hline
    SUMME \emph{inkl. Gewichtung}                 & \textbf{100 \%}     & \textbf{2.95}       & \textbf{2.55}       \\
    \hline
  \end{tabular}
\end{table}

\newpage

\section{Autorisierung}

Damit ein Benutzer mit der Rolle \enquote{candidate} nur Ressourcen einsehen und bearbeiten kann, zu denen dieser berechtigt ist,
wird ein System für eine Autorisierung benötigt. Diese Autorisierungsbibliothek sollte mit dem Authentifizierungs gem \emph{Devise} kompatibel sein, da dieses in dieser PA eingesetzt wird.
Ausserdem sollte die Implementation möglichst einfach mit den in \ref{} beschriebenen Testmitteln testbar sein.
Auch für diesen Variantenvergleich wurden zwei der beliebtesten Bibliotheken herbeigezogen, welche den genannten Kriterien entsprechen.

\subsubsection{Variante 1: \enquote{CanCanCan}}

Alle Berechtigungen können in \emph{CanCanCan} in einer einzigen \enquote{Fähigkeitsdatei} definiert werden, sodass die Berechtigungslogik für einfache Wartung und Tests an einer Stelle bleibt.
Ausserdem stellt die Bibliothek viele Helper zur Verfügung, um diese Berechtigungen in verschiedensten orten der Applikation abzufragen. Auch wird das Testing mittels \emph{RSpec} durch mitgelieferte \enquote{Matcher} erheblich erleichtert.

\subsubsection{Variante 2: \enquote{Pundit}}

\emph{Pundit} definiert sogenannte \enquote{Policies} für einzelne Ruby-Klassen. Das heisst, es gibt nicht nur eine einzelne \enquote{Fähigkeitsdatei},
sondern die Berechtigungen werden in mehrere Dateien aufgeteilt und explizit für jedes Model definiert. Jedoch können auch globale Policies definiert werden.
Auch stehen zwar sehr ähnliche, jedoch nicht so viele Helper-Klassen zur verfügung wie in dem \emph{CanCanCan} gem. Auch \emph{RSpec Matcher} kommen nicht out-of-the-box mit dem gem mitgeliefert.

\subsubsection{Nutzwertanalyse}

Der Gewinner der folgenden Nutzwertanalyse ist eindeutig die 1. Variante. Die beiden gems unterscheiden sich kaum durch deren Umfang an Features und schlussendlich waren vor allem die Vorkenntnisse
ausschlaggebend für diese Entscheidung. Gearbeitet hat der Kandidat bereits mit beiden gems, jedoch kommt \emph{CanCanCan} in Projekten der Renuo AG wesentlich häufiger zum Einsatz.

\begin{table}[H]
  \rowcolors{2}{gray!10}{white}
  \begin{tabular}{|l|l|l|l|}
    \hline
    \rowcolor{PrimaryColor!30} \textbf{Kriterium} & \textbf{Gewichtung} & \textbf{Variante 1} & \textbf{Variante 2} \\
    \hline
    Kenntnisse                                    & 40 \%               & 5                   & 2                   \\
    \hline
    Dokumentationen/Anleitungen                   & 20 \%               & 5                   & 5                   \\
    \hline
    Komplexität                                   & 10 \%               & 4                   & 3                   \\
    \hline
    Umfang/Features                               & 15 \%               & 3                   & 4                   \\
    \hline
    Integration in das Framework                  & 15 \%               & 5                   & 5                   \\
    \hline
    \hline
    SUMME \emph{inkl. Gewichtung}                 & \textbf{100 \%}     & \textbf{4.6}        & \textbf{3.45}       \\
    \hline
  \end{tabular}
\end{table}
