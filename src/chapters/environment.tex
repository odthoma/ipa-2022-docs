\chapter{Umfeld} \label{ch:environment}

\section{Arbeitsplatz}

Die IPA wird grösstenteils vor Ort in der Renuo AG in Wallisellen durchgeführt werden.
Dort stehen alle Mittel zur Verfügung, um diese ordnungsgemäss durchzuführen.
Dennoch wird es aufgrund des langen Arbeitsweges des Kandidaten vereinzelte Homeoffice-Tage geben. Da dies
bereits im Arbeitsalltag des Kandidaten liegt, ist auch dieser Arbeitsplatz gut Ausgestattet.
Ausserdem wurden zwei Kopien des PkOrg Standardkriterienkatalog ausgedruckt und auf die Arbeitsplätze gelegt,
um so jederzeit direkt darauf zugreifen zu können.

\section{Firmenstandards}

Für diese PA gelten die Firmenstandards aus dem \emph{Renuo Application-Setup-Guide}. Diese sind
dem Kandidaten bereits aus seinem Arbeitsalltag bekannt. Es wird eine Testabdeckung von 100\% mit nur wenigen begründeten Ausnahmen erwartet
und das Rubocop-Styleguide soll eingehalten werden. Diese Anforderungen werden durch eine CI/CD Pipeline kontrolliert und durchgesetzt.

\section{Eingesetzte Tools} \label{sec:tools}

Ein modernes MacBook Pro mit der neusten macOS Version (\emph{12.4 Monterey}) wird für den gesamten Entwicklungsprozess und die Dokumentation eingesetzt.
Während Homeoffice-Tagen würde zudem eine Arch-Linux Maschine zum Einsatz kommen.

\subsection{Dokumentation und Projektmanagement}

\begin{itemize}
    \item Um die Arbeitszeiten des Kandidaten einzutragen, wird Redmine mit dem Tracky Plugin verwendet.
    \item Der IPA-Bericht wird in LaTeX verfasst. Dazu verwendet der Kandidat Visual Studio Code mit diversen Erweiterungen
          und versioniert diese über ein Git-Repository. Der IPA-Bericht baut auf der Vorlage \cite{Buhler_ipa-template_2022} auf.
    \item Sowohl UMLet als auch das textbasierte Tool mermaid werden für UML-Diagramme verwendet.
\end{itemize}

\subsection{Entwicklung}

\begin{itemize}
    \item Die (Haupt-) Entwicklungsumgebung, die zum Einsatz kommt, ist JetBrains RubyMine.
    \item Fork wird als grafischer Git-Client eingesetzt.
    \item iTerm \& Alacritty als Terminal
    \item Der Kandidat verwendet sowohl Firefox als auch einen Chromium-basierten Browser während der Entwicklungs- und Testphase,
          um so eine Kompatibilität über so ziemlich alle modernen Browser sicherzustellen.
\end{itemize}
