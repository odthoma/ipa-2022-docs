% see B6.2a
\chapter{Projektaufbauorganisation}

Im Unterschied zur üblichen Arbeit im Betrieb werden zwei Experten die Arbeit des Kandidaten Begleiten und Bewerten.
Die Projektaufbauorganisation ist in \ref{fig:organigram} visualisiert.

\vspace*{0.5cm}

\begin{figure}[H]
  \begin{multicols}{2}
    \begin{forest}
      for tree={draw,grow'=0,folder,align=left}
      [\textbf{\varCompany} \emph{(Durchführungsort)}
        [\textbf{\varCompanyDepartment}
          [(VF) \\ \varResponsibleSpecialist]
          [(K) \\ \varCandidate]
        ]
      ]
    \end{forest}

    \begin{forest}
      for tree={draw,grow'=0,folder,align=left}
      [\textbf{\varExaminationBoard}
        [\textbf{\varExaminationBoardDepartment}
          [(HEX) \\ \varPrimaryExpert]
          [(NEX) \\ \varSecondaryExpert]
        ]
      ]
    \end{forest}
  \end{multicols}
  \begin{center}
    \begin{forest}
      for tree={draw,grow'=0,folder,align=left}
      [\textbf{Kantonsschule Büelrain Winterthur}
        [\textbf{Informatikmittelschule (IMS)}
          [(BB) \\ \varVocationalTrainer]
        ]
      ]
    \end{forest}
  \end{center}
  \caption[\enquote{Organigramm der am Projekt teilnehmenden Personen} visualisiert mit TikZ Forest]{\gls{Organigramm} der am Projekt teilnehmenden Personen}
  \label{fig:organigram}
\end{figure}
