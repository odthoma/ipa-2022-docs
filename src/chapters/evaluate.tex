\chapter{Auswerten}

\section{Reflexion}

Das Projekt konnte durch eine saubere Planung und ein gutes Engagement den Anforderungen entsprechend umgesetzt werden.
Der Kandidat war kaum auf die Hilfe anderer Fachpersonen angewiesen und konnte dieses relativ selbstständig realisieren. 
 Wie in jedem Projekt, gab es auch hier gewisse Stresssituationen und Probleme, welche allesamt rasch und professionell bewältigt werden konnten. Die vorgegebene Zeit wurde eingehalten und es wurden nur wenige Stunden ausserhalb der regulären Arbeitszeit in die IPA investiert.

\section{Persönliche Bilanz}

Nach diesen zehn Tagen harter Arbeit bin ich sehr zufrieden mit dem Endergebnis, welches meine Erwartungen stark übertroffen hat.
In der Zeit vor meiner IPA konnte ich relativ wenig mit dem Ruby on Rails Framework arbeiten, weshalb es mir auch um so mehr Spass gemacht hat,
den Assessment-Assistant umzusetzen. Auch der Gedanke, dass die Applikation bald für meine Nachfolger eingesetzt werden soll und somit einen richtigen
Nutzen hat (und nicht nur eine Schul-Arbeit ist), ist sehr zufriedenstellend. Dokumentieren und Texte schreiben sind definitiv nicht meine Stärke, dennoch bin ich der Meinung, einen
im Grossen und Ganzen guten Bericht verfasst zu haben.

\subsection{Erfolge}
Auch wenn es eine echte Herausforderung für mich war und ich dadurch teilweise auch etwas Zeit verloren habe, bin ich 
sehr froh darüber, den IPA-Bericht in LaTeX verfasst zu haben. Ich habe eine für mich komplett neue Welt kennengelernt und konnte trotz anfänglichen Schwierigkeiten
einen sauberen Bericht abliefern und meine Ziele erreichen.

Während meiner bisherigen Zeit in der Renuo AG habe ich kaum testgetrieben entwickelt. Daher ist es ein sehr grosser Erfolg für mich, 
diese Arbeitsweise in diesem Projekt so gut es geht umgesetzt zu haben.

\subsection{Probleme}
Das Einhalten des Zeitplans gegen das Ende der PA war knifflig und das Projekt hat sich für die gegebene
Zeit als sehr umfangreich herausgestellt und wurde von mir unterschätzt. Daher musste ich ein wenig Freizeit aufwenden,
um den IPA-Bericht aufzuholen. Die einzelnen Aufgaben in der Realisierungsphase hätte ich noch feiner aufteilen und schätzen müssen, um mir so einen
besseren Überblick über den Projektstand zu verschaffen.

Auch das Abwickeln des Projektes nach der \emph{IPERA} Projektmanagementmethode war für mich eine Herausforderung und war schwerer als gedacht. 
Ich war eine solche Herangehensweise an die Umsetzung von Softwareprojekten einfach nicht gewohnt. In meinem
Berufsalltag arbeite ich agil und erstelle keine vollständigen Planungen im Voraus. Auch wenn sich \emph{IPERKA} für eine PA in dieser Grössenordnung
eignet, würde ich beim nächsten Mal auf eine andere Methode zurückgreifen.

\section{Mögliche Erweiterungen}

Während der Realisierungsphase hatte ich Schwierigkeiten, nicht zu stark von den Anforderungen abzuschweifen und somit Features zu
entwickeln, die eigentlich gar nicht gefordert waren. Auch wenn das Projekt in einem funktionierenden Zustand ist, gibt es viele Möglichkeiten
das Projekt zu erweitern. In diesem Abschnitt habe ich einige meiner Ideen kurz zusammengefasst:

\begin{itemize}
  \item Die Korrektur-Kommentare sollten editierbar gemacht werden - momentan lassen sich diese nur erstellen. Bei einem schreibfehler ist dies sehr ärgerlich.
  \item Eine Bewertung pro abgegebener Lösung anhand eines einfachen Punkesystems könnte implementiert werden. So erhält der Korrektor nach der Korrektur einen schnellen Überblick darüber,
        wie gut die Bewerber abgeschnitten haben.
  \item Wenn das Assessment durch den Background-Job beendet wird, wird dies nicht direkt auf dem Bildschirm der Bewerber angezeit. Sprich, diese sehen
        noch das Upload-Formular, obwohl dieses zu diesem Zeitpunkt nicht mehr funktioniert.
        Um die Applikation nicht nur hier, sondern an verschiedensten Orten interaktiver und reaktiver zu gestalten, könnten mehr TurboStreams (d.h. WebSockets) eingesetzt werden.
  \item Beim Erfassen der Aufgaben durch den Betreuer sollte es möglich sein, eine ungefähre Zeitschätzung zu machen. Dies sollte dem Kandidaten helfen,
        seine Zeit besser einzuteilen. Dadurch könnte ausserdem die Gesamtzeit für das Assessment berechnet werden.
  \item Lösungen sollten zusätzlich zu den File-Uploads ebenfalls als Rich-Text abgegeben werden. In manchen Fällen ist das einfach praktischer.
\end{itemize}
