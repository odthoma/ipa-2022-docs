% See B1
\chapter{Kurzfassung}

Die Kurzfassung gibt einen Überblick über das vorliegende Projekt.

\section{Ausgangssituation}

Die Renuo AG nimmt nun schon seit einigen Jahren regelmässig neue Praktikanten aus der Informatikmittelschule (IMS) bei sich auf, die
dort das letzte Jahr ihrer Ausbildung zum Softwareentwickler absolvieren. Die besten acht bis zehn Bewerber haben dann die Chance, ein Assessment
vor Ort in der Firma zu absolvieren. Dieses besteht aus zwölf gemischten Programmieraufgaben und dauert etwa vier Stunden.
Die Lösungen aller Kandidaten werden einzeln ausgewertet und es folgt ein persönliches Gespräch, um die Aufgaben miteinander zu Besprechen.
Aufgrund dieser Kriterien wird dann entschieden, wer ein für das einjährige Praktikum in der Renuo AG antreten kann.
Jeder Kandidat gibt seine Lösungen auf eine andere Art und Weise ab; sei es per USB-Stick, per E-Mail oder sogar auch auf Papier.
Ausserdem wird \emph{jeder} Bewerber über seine Lösungen und Bewerbungsstatus informiert.
Dies beansprucht sehr viel Zeit und stellt einen grossen Aufwand dar. Erhofft wird durch diese PA eine Vereinfachung und Vereinheitlichung des Assessment-Prozesses in der Renuo AG.

\section{Umsetzung}

\section{Ergebnis}
