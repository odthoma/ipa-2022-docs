% See B1
\chapter{Kurzfassung}

Die Kurzfassung gibt einen Überblick über das vorliegende Projekt.

\section{Ausgangssituation}

Die Renuo AG nimmt nun schon seit einigen Jahren regelmässig neue Praktikanten aus der Informatikmittelschule (IMS) bei sich auf, die
dort das letzte Jahr ihrer Ausbildung zum Softwareentwickler absolvieren. Die jeweils acht bis zehn besten Bewerber haben die Chance, ein Assessment
vor Ort in der Firma zu absolvieren. Dieses besteht aus zwölf gemischten Programmieraufgaben und dauert etwa vier Stunden.
Die Lösungen aller Kandidaten werden einzeln ausgewertet und es folgt ein persönliches Gespräch, um die Aufgaben miteinander zu Besprechen.
Aufgrund dieser Kriterien wird dann entschieden, wer ein für das einjährige Praktikum in der Renuo AG antreten kann.
Jeder Kandidat gibt seine Lösungen auf eine andere Art und Weise ab; sei es per USB-Stick, per E-Mail oder sogar auch auf Papier.
Ausserdem wird jeder Bewerber über seine Lösungen und Bewerbungsstatus informiert.
Dies beansprucht sehr viel Zeit und stellt einen grossen Aufwand dar. Erhofft wird durch diese PA eine Vereinfachung und Vereinheitlichung des Assessment-Prozesses in der Renuo AG.

\section{Umsetzung}

Dieses Projekt wurde mit der Projektmanagementmethode \emph{IPERKA} abgewickelt. In den ersten Phasen der PA wurde die Aufgabenstellung analysiert
und durch den Einsatz von Modellen und Konzepten wurden verschiedene Lösungsansätze aufgezeigt und Entscheidungen getroffen. 

Die Software wurde dann in der Realisierungsphase mit Ruby on Rails, JavaScript, HTML und CSS umgesetzt und wird auf einem Heroku Cloud-Server gehostet.
Es wurde strikt nach \enquote{Renuo-Standards} gearbeitet und es wurden bekannte Softwarebibliotheken eingesetzt,  
um gewisse Funktionalitäten möglichst effizient und fehlerfrei zu implementieren. Durch ein gut durchdachtes Testkonzept wurde testgetrieben entwickelt 
und das System laufend auf seine Richtigkeit überprüft. Dadurch konnte eine ständige Testabdeckung von 100\% sichergestellt werden.

\section{Ergebnis}

Das Endergebnis entspricht vollständig den Anforderungen aus \ref{ch:task} und befindet sich in einem Funktionsfähigen zustand.
Assessments können nun auf der Plattform erfasst, durchgeführt und ausgewertet werden. Die Bewerber können ihre Lösungen auf einen Amazon S3
Storage-Server hochladen und erhalten automatisierte E-Mail Benachrichtigungen über Sparkpost.
