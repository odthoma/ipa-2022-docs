% [2]: Seite 11
\chapter{Aufgabenstellung}

Dieses Kapitel beinhaltet die komplette Aufgabenstellung im originalen Wortlaut von PkOrg.ch.

\section{Ausgangslage}

Der Einstellungsprozess der Renuo ist zeitaufwendig wegen der intensiven Assessments, welche Bewerber
durchlaufen müssen. So unterziehen sich z.B. IMS-Praktikanten (8 Assessments bei etwa 40 Bewerbern) einer
vierstündigen Session mit 12 Programmieraufgaben. Diese Aufgaben werden gewissenhaft korrigiert. Die
Resultate werden den Bewerbern mitgeteilt, sodass die investierte Zeit bei jedem Ausgang des Assessments
nicht vergebens ist.

Diese IPA hat zum Ziel die Koordination des Renuo-Assessment-Prozesses mit einer Web-Plattform zu
erleichtern. Das Austeilen der Aufgaben und die rechtzeitige Abgabe der Lösungen soll vereinheitlicht werden.
Im Weiteren sollen die Aufgaben neu direkt in der Plattform erfasst und die Lösungen ausgewertet werden
können.

\section{Detaillierte Aufgabenstellung}

Die in dieser Arbeit entwickelte Web-Plattform vereinheitlicht und erleichtert den Renuo-Assessment-Prozess.

\textbf{Die Software bedient folgende Benutzerrollen:}
\begin{itemize}
    \item \enquote{Bewerber}: durchläuft ein Assessment
    \item \enquote{Betreuer}: koordiniert den Ablauf des Assessments
    \item \enquote{Korrektor}: korrigiert die Lösungen
\end{itemize}

\emph{(Die Rollen des Betreuers und des Korrektors müssen nicht separat gehandhabt werden.)}

\newpage

\subsection{Funktionale Anforderungen}

\begin{itemize}
    \item F1: \enquote{Als Bewerber kann ich per Einladungs-Link an einer Assessment-Session teilnehmen} (damit ich Zugriff auf die Aufgaben erhalte.)
    \item F2: \enquote{Als Bewerber sehe ich, wie lange die Assessment-Session dauert, damit ich mir meine Arbeit richtig aufteilen kann.}
    \item F3: \enquote{Als Bewerber kann ich meine Lösungen zu jeder Aufgabe einzeln hochladen damit meine Leistung mit den der anderen Bewerber verglichen werden kann.}
          \\Die Aufgaben können wie folgt abgegeben werden:
          \begin{itemize}
              \item als Dateien (z.B. Fotos von handgezeichneten Diagrammen)
          \end{itemize}
    \item F4: \enquote{Als Bewerber erhalte ich nach dem Assessment Einsicht in den Korrektur-Kommentar.}
    \item F5: \enquote{Als Betreuer kann ich Assessment-Sessionen konfigurieren. So kontrolliere ich wann und wie lange das Assessment stattfindet und wer daran teilnimmt.}
          \begin{itemize}
              \item Bewerber werden per E-Mail-Adresse erfasst.
              \item Die Assessment-Session durchläuft folgende Status: \emph{preparing, open, reviewing, closed}
              \item Es können beliebig viele Assessment-Sessionen laufen - auch gleichzeitig.
          \end{itemize}
    \item F6: \enquote{Als Betreuer kann ich zu einer Assessment-Sessionen Aufgaben mit einer Beschreibung verwalten.}
          \begin{itemize}
              \item Die Beschreibung kann formatiert werden (HTML / Markdown)
              \item Als Bewerber kann ich die Aufgabenstellung im Portal einsehen
          \end{itemize}
    \item F7: \enquote{Als Betreuer kann ich Bewerber per Instruktions-E-Mail mit Invite-Link zu einer Assessment-Session einladen. So stelle ich sicher, dass die Bewerber rechtzeitig informiert sind und sich auf die Session vorbereiten können.}
          \begin{itemize}
              \item Die E-Mail soll die E-Mail-Adresse des Betreuers im reply-to-Header enthalten.
              \item Der E-Mail-Inhalt darf hardcoded sein
          \end{itemize}
    \item F8: \enquote{Als Korrektor kann ich pro Assessment-Session jede Aufgabe mit den dazu abgegebenen Lösungen einsehen und Korrektur-Kommentare abgeben.}
          \begin{itemize}
              \item Die Korrektur findet pro Aufgabe und nicht pro Person statt, damit ein Vergleich zwischen den Kandidaten möglich ist
              \item Die Reihenfolge der Lösungen ist bei jeder Aufgabe zufällig
          \end{itemize}
    \item F9: \enquote{Als Korrektor sehe ich während der Korrektur nie, zu wem (E-Mail-Adresse) die abgegebene Lösung gehört. So bleibe ich unparteiisch.}
    \item F10: \enquote{Als Korrektor kann ich die Assessment-Session schliessen um den Bewerbungsprozess zu einer Entscheidung zu bringen.}
          \begin{itemize}
              \item Das generiert eine Tabelle mit allen Korrektur-Kommentaren pro Person.
              \item Der Bewerber kann seine Resultate auf dem Portal einsehen
          \end{itemize}
\end{itemize}

\newpage

\subsection{Nicht-funktionale Anforderungen}

\begin{itemize}
    \item N1 Software-Design: Die Applikation ist für Heroku (12factor) mit AWS S3-Speicher konzipiert.
    \item N2 Software-Design: Die Applikation ist versendet E-Mails via Sparkpost.
    \item N3 Sicherheit: Bewerber dürfen die Lösungen anderer Bewerber nicht sehen können. Die Öffentlichkeit hat niemals Zugriff auf Aufgaben oder Lösungen.
    \item N4 Performance: Das Hochladen von (auch grossen) Dateien darf den Betrieb für Bewerber nicht beeinträchtigen.
    \item N5 UX: Für die Bedienung darf kein dediziertes Benutzerhandbuch notwendig sein.
    \item N6 Dokumentation: Wichtige Teile der Software müssen mit den richtigen UML-Diagrammen beschrieben werden:
          \begin{itemize}
              \item Entity-Relation-Diagramm für die Business-Domain
              \item State-Diagramm (Zustandsdiagramm) zur Kontrolle des Assessment-Ablaufs
          \end{itemize}
    \item N7 Tests: Der Happy-Path für alle Rollen muss automatisiert getestet sein (UI-Tests).
    \item N8 Tests: Die Unit-Test-Abdeckung des geschriebenen Codes beträgt 100%.
\end{itemize}

\section{Mittel und Methoden}

\textbf{Für die IPA werden folgende Technologien eingesetzt:}
\begin{itemize}
    \item Ruby on Rails
    \item HTML5 (HTML, JS, CSS)
\end{itemize}
