\section{Umsetzung der Datenstruktur}

Die grundlegende Datenstruktur wurde anhand der vordefinierten Rails-Generator Commands \ref{subsec:model-generation} aufgebaut.
Damit war der grösste Teil der Arbeit bereits erledigt und es mussten lediglich die Assoziationen in den jeweiligen Model-Klassen ergänzt werden.
Diese werden dort durch die ActiveRecord Funktionen \mintinline{ruby}{has_many}, \mintinline{ruby}{has_one}, und \mintinline{ruby}{belongs_to} abgebildet.

Zwischen der \emph{user} und \emph{assessment} Tabelle herrscht eine many-to-many Assoziation, da ein Bewerber mehrere Assessments haben kann, 
woran wiederum mehrere Bewerber teilnehmen können. Dafür wird die Join-Tabelle \emph{assessment\_participations} eingesetzt. 
Um nun sicherzustellen, dass ein Bewerber nur einmal in einer Assessment-Instanz vorkommen kann, wurde über eine ActiveRecord Datenbankmigration ein 
unique composite Index über beide Fremdschlüssel erstellt.

\begin{codebox}
\begin{minted}{ruby}
def change
  add_index :assessment_participations, %i[user_id assessment_id], unique: true
end

\end{minted}
\end{codebox}

Ausserdem sollte ein Bewerber nur eine einzige Lösung pro Aufgabe abgeben können. Dafür wurde auf der \emph{solutions}
Tabelle ebenfalls ein unique composite Index erstellt. Beide Indizes werden auch auf Application-Level in den entsprechenden Model-Klassen
durch Validierungen abgesichert, wodurch lesbarere Error-Messages für die Formulare im Frontend generiert werden können.

\begin{codebox}
\begin{minted}{ruby}
def change
  add_index :solutions, %i[user_id task_id], unique: true
end
\end{minted}
\end{codebox}

\begin{codebox}
\begin{minted}{ruby}
class Solution < ApplicationRecord
  belongs_to :task
  belongs_to :user
  has_many :comments, dependent: :destroy

  validates :user_id, uniqueness: { scope: :task_id }

  ...
end
\end{minted}
\end{codebox}

\begin{codebox}
\begin{minted}{ruby}
class AssessmentParticipation < ApplicationRecord
  belongs_to :assessment
  belongs_to :user

  validates :user_id, uniqueness: { scope: :assessment_id }

  ...
end
\end{minted}
\end{codebox}

\newpage

\section{CRUDs}

Bei Ruby on Rails handelt es sich um ein \gls{mvc} Framework. Daher wurde für alle generierten Model-Klassen
auch eine Controller-Klasse und die entsprechenden HTML-Views erstellt. Diese folgen grösstenteils dem Rails-Standard \cite{default_controller_actions} und beinhalten einfache \gls{crud} Operationen.
Basierend auf der Aufgabenstellung wurden nicht für jede Ressource alle vier Operationen implementiert.

Die Funktionen in einem Rails ActionController werden \enquote{Actions} genannt. Der folgende Codeausschnitt zeigt zwei dieser Actions,
die die Erstellung eines Assessments implementieren. Ein wichtiges Konzept hierbei sind die \emph{ActionController::StrongParameters}.
Durch diese werden die Anzahl der Felder, die bei der Erstellung eines Assessments mitgegeben werden können, limitiert. Auch dieses
Sicherheitskonzept wurde in allen Controllern implementiert.

\begin{codebox}
\begin{minted}{ruby}
class AssessmentsController < ApplicationController
  def new # GET /assessments/new
    @assessment = Assessment.new
  end

  def create # POST /assessments
    @assessment = Assessment.new(assessment_params)
    if @assessment.save
      redirect_to assessment_path(@assessment)
    else
      render :new, status: :unprocessable_entity
    end
  end

  ...

  private

  def assessment_params
    params.require(:assessment).permit(:title, :starts_at, ...)
  end
end
\end{minted}
\end{codebox}

Nachdem eine Action durchlaufen ist und kein HTML-Template explizit gerendert wurde, sucht sich das Framework automatisch
ein Template mit dem entsprechenden Namen und rendert dieses. So wird beispielsweise in der 
oberhalb gezeigten \emph{new} Action eine Instanzvariable gesetzt und dann automatisch das \emph{views/assessments/new.html.slim} HTML-Template gerendert.
Diese Instanzvariable kann dann in den Templates verwendet werden.

So wurden für alle notwendigen Models, sprich \emph{Assessment}, \emph{Task}, \emph{Solution} und \emph{Comment} mit der Hilfe des \enquote{simple\_form} Gems
die notwendigen Formulare erstellt.

\begin{codebox}
\begin{minted}{ruby}
= simple_form_for @assessment do |f|
    = f.input :title
    = f.input :starts_at
    = f.button :submit
\end{minted}
\end{codebox}
