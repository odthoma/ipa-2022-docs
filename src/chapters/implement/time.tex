\section{Zeitbegrenzung}

Nachdem das Assessment durch den Admin gestartet wurde und es seinen Zustand zu \enquote{open} gewechselt hat, wird
ein Background-Job erstellt, der das Assessment rechtzeitig beenden soll. Dieser wird auf der Produktions-Umgebung 
über das Sidekiq ActiveJob-Backend in die Warteschlange einer Redis in-memory Datenbank gestellt. 

Der Job ist sehr simpel und ruft eine einzige Funktion auf der Assessment-Instanz auf, 
die über die State-Machine dessen Zustand wechselt. 

\begin{codebox}
\begin{minted}{ruby}
class EndAssessmentJob < ApplicationJob
  queue_as :default

  def perform(assessment)
      assessment.finish!
  end
end
\end{minted}
\end{codebox}

Ausserdem soll für den Bewerber jederzeit ersichtlich sein, wie lange dieser für das Assessment Zeit hat. 
Dafür wird ein JS basierter Countdown über das Stimulus Framework implementiert, der die verbleibende Zeit runterzählt.

Der Stimulus-Controller wird hier über das \emph{data-controller} HTML-Attribut an das \mintinline{html}{<h3>} Element gebunden.
Ausserdem wird das exakte DateTime, an dem das Assessment beendet wird, im Unix-Timestamp-Format an die Controller-Klasse über ein \emph{data-value} Attribut weitergegeben.

\begin{codebox}
\begin{minted}{html}
h3(data-controller='assessment-time' data-assessment-time-end-value=@assessment.ends_at.to_i)
\end{minted}
\end{codebox}

Der Timestamp kann dann im Controller als Stimulus-Value verwendet werden. Dort hat man Zugriff auf dieses Value über
\mintinline{javascript}{this.endValue}.

\begin{codebox}
\begin{minted}{javascript}
static values = { end: Number }
  
\end{minted}
\end{codebox}

Im Stimulus-Controller wird die verbleibende Zeit aus der Differenz zwischen der Endzeit und der aktuellen Zeit berechnet.
Stimulus bietet gewisse Lifecycle-Callbacks wie \mintinline{javascript}{connect()} und \mintinline{javascript}{disconnect()},
in denen jeweils ein Intervall gestartet oder gestoppt wird. Der Textinhalt vom \mintinline{html}{<h3>} Element, mit dem der Controller
verbunden ist, wird dann einfach sekündlich durch die berechnete Zeit ersetzt.

\begin{codebox}
\begin{minted}{javascript}
connect () {
  this.countdownInterval = setInterval(() => {
      this.element.innerText = this.parsedTime;
    }, 1000);
}

disconnect () {
  clearInterval(this.countdownInterval);
}
\end{minted}
\end{codebox}
