\section{Formatierte Textinhalte}

Die Aufgaben und Korrektur-Kommentare sollen durch formatierten Text strukturiert dargestellt werden.
Dazu wird die bereits angesprochene \enquote{ActionText} Bibliothek verwendet, die ebenfalls standardmässig mit Ruby on Rails mitgeliefert wird.
Diese enthält neben einigen praktischen Helpern auch den JS Trix-Editor, der alles von der Formatierung über Links, Zitate und Listen bis hin zu eingebetteten Bildern übernimmt. 

Der vom Trix-Editor generierte Rich-Text-Inhalt wird im HTML-Format gespeichert und eingebettete Bilder (oder andere Anhänge) werden automatisch durch ActiveStorage verwaltet.

Über den generator-command \mintinline{bash}{rails g action_text:install} werden die notwendigen migrationen erstellt und
der Trix-Editor wird über den yarn JS Package-Manager installiert. Dann wurde lediglich die \mintinline{ruby}{has_rich_text} Funktion
zu dem Model hinzugefügt. 

\begin{codebox}
\begin{minted}{ruby}
class Task < ApplicationRecord
  has_rich_text :body
  
  ...
end
\end{minted}
\end{codebox}

\begin{codebox}
\begin{minted}{ruby}
= simple_form_for [@assessment, @task] do |f|
    = f.input :body, as: :rich_text_area
    = f.button :submit
\end{minted}
\end{codebox}

Rich-Text Inhalte wurden sowohl für die Aufgabenbeschreibung als auch die Korrektur-Kommentare umgesetzt und sehen
in den Formularen so aus:

\begin{figure}[H]
  \centering
  \includegraphics[width=14cm]{images/trix.png}
  \caption{Der ActionText \emph{trix} Rich-Text-Editor}
\end{figure}
