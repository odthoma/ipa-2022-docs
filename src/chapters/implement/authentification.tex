
\section{Authentifizierung}

Als Authentifizierungssystem wird das \enquote{Devise} gem eingesetzt. Dies ist eine sehr bekannte, vollumfängliche Lösung für Ruby on Rails,
welche einem die ganze Arbeit abnimmt, ein solches System selbst zu implementieren. Durch die mitgelieferten Generator-Commands
kann schnell eine stabile Grundstruktur geschaffen werden - sogar inklusive allen HTML-Templates, welche nur noch leicht angepasst werden mussten. 
Die notwendigen Model-Klassen und Datenbankmigrationen werden so ebenfalls vollautomatisch generiert.

\begin{codebox}
\begin{minted}{shell}
rails generate devise:install
rails generate devise User
rails generate devise:views

rails db:migrate
\end{minted}
\end{codebox}

Devise basiert auf sogenannten \enquote{Strategies}, welche im User-Model definiert werden. Diese beschreiben,
welche Funktionalitäten von Devise aktiviert werden sollen und welche nicht.

\begin{codebox}
\begin{minted}{ruby}
class User < ApplicationRecord
  devise :database_authenticatable, :recoverable, :trackable
          :confirmable, :rememberable, :validatable, :registerable
end
\end{minted}
\end{codebox}

Hier wird auch die \mintinline{ruby}{:registerable} Strategy aufgelistet, die einen Registrierungsprozess implementiert.
Und das, obwohl keine selbstständigen Registrierungen möglich sein sollten, da Bewerber nur per Einladungs-Link an einem Assessment teilnehmen können. 
Die Strategy wird hier allerdings bewusst eingesetzt, da sie auch ein Formular für das Ändern von E-Mail-Adresse und Passwort bietet. 
Das eigentliche Registrierungsformular \emph{/user/registrations/new} wird dann einfach explizit über die Devise route-definition deaktiviert, während
die edit-routes manuell eingetragen werden.

\begin{codebox}
\begin{minted}{ruby}
Rails.application.routes.draw do
  devise_for :users, skip: %i[registrations]

  devise_scope :user do
    get 'users/edit', to: 'devise/registrations#edit'
    put 'users', to: 'devise/registrations#update'
  end

  ...
end
\end{minted}
\end{codebox}

Ausserdem werden Strategies eingesetzt um:
\begin{enumerate}
  \item Passwörter via E-Mail zurückzusetzen
  \item Jeden login/logout zu protokollieren inkl. Zeit, Datum, IP-Adresse
  \item Formulareingaben zu validieren
  \item Nach einer Änderung der E-Mail-Adresse diese via Link zu bestätigen
\end{enumerate}
