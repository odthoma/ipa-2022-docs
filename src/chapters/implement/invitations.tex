\section{Assessment-Einladungen}

\subsection{Hinzufügen zum Assessment}

Durch den Admin sollen Bewerber per E-Mail-Adresse zu einem Assessment eingeladen werden können. Dazu wird
das \enquote{devise\_invitable} Gem verwendet, welches perfekt mit dem verwendeten Authentifizierungssystem zusammenspielt.

Durch die Installation werden automatisch die benötigten Felder auf der \emph{user} Tabelle durch eine ActiveRecord Datenbankmigration 
erstellt. Dann kann die Bibliothek wie eine Devise-Strategy gehandhabt werden:

\begin{codebox}
\begin{minted}{ruby}
class User < ApplicationRecord
  devise :invitable, ...
end
\end{minted}
\end{codebox}

Das Gem dient lediglich als eine Grundlage und muss für den geforderten Use-Case entsprechend angepasst werden.
Denn standardmässig sind nur \enquote{globale} Einladungen möglich.

Es gibt zwei Fälle die eintreten können:
\begin{itemize}
  \item Der Benutzer existiert noch nicht und muss sich erst noch einen neuen Account erstellen
  \item Der Benutzer hat schonmal an einem Assessment teilgenommen und existiert bereits
\end{itemize}

Es wurde ein eigener \emph{invitations\_controller.rb} geschrieben, dessen \emph{create} Action zuerst überprüft, ob ein Benutzer
mit der angegebenen E-Mail-Adresse bereits in der Datenbank existiert. Wenn nicht, wird über die devise\_invitable Funtkion \mintinline{ruby}{User.invite!} 
ein neuer Benutzer erstellt und automatisch ein \emph{invitation\_token} hinterlegt. Über einen Link, der diesen Token als GET-Parmaeter beinhaltet,
kann der Benutzer sich später selbstständig ein Passwort setzen.

\begin{codebox}
\begin{minted}{ruby}
@user = User.find_by(email: params[:email]) ||
        User.invite!({ email: params[:email], skip_invitation: true }, current_user)
@user.invite_to(@assessment)
\end{minted}
\end{codebox}

\begin{codebox}
\begin{minted}{ruby}
def invite_to(assessment)
  errors.add(:email, :already_in_assessment) and return if assessment.candidates.exists?(id)
  
  assessments << assessment
  send_assessment_invitation(assessment)
end
\end{minted}
\end{codebox}

Im Anschluss wird das Assessment über den Shovel-Operator \mintinline{ruby}{<<} zu den Assessments des Benutzers
hinzugefügt und eine E-Mail wird versendet.

\newpage

\subsection{E-Mail Benachrichtigungen}

Die besagte Einladungs-E-Mail wird über eine ActionMailer Klasse versendet.

\begin{codebox}
\begin{minted}{ruby}
def assessment_invitation(recipient, assessment, invitation_token)
  @invitation_token = invitation_token
  @recipient = recipient
  @assessment = assessment

  mail(to: @recipient.email,
        subject: t('invitation_mailer.assessment_invitation.subject'),
        reply_to: User.where(role: User.roles[:admin]).pluck(:email))
end
\end{minted}
\end{codebox}

Das E-Mail Template beinhaltet neben den generellen Informationen zu dem Assessment auch einen Invite-Link, über den
sich der eingeladene Benutzer einen Account erstellen kann. Dieser wird allerdings nur gerendert, wenn noch kein Account existiert.
So kann das selbe E-Mail für beide Fälle verwendet werden.

\begin{codebox}
\begin{minted}{ruby}
- if @invitation_token.present?
  p = link_to t('devise.mailer.invitation_instructions.accept'),
      accept_user_invitation_url(invitation_token: @invitation_token)
\end{minted}
\end{codebox}
