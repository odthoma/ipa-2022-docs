\section{Tag 1 - 23.05.2022}

\begin{tabularx}{\textwidth}[H]{|c|X|}
  \hline
  Erledigte Arbeiten         &
  \textbf{Vormittag:}
  \begin{itemize}
    \item Start mit der LaTeX Dokumentation
    \item Übertragen der Aufgabenstellung von PkOrg
    \item Erstellen des Zeitplans
  \end{itemize}
  \textbf{Nachmittag:}
  \begin{itemize}
    \item Projektaufbauorganisation darstellen
    \item Ausgangssituation in der Kurzfassung formulieren
    \item Arbeitspakete nach \emph{IPERKA} definieren
    \item Systemschnittstellen darstellen und mit der Beschreibung starten.
  \end{itemize}
  \\ \hline

  Erreichte Ziele            &
  \begin{itemize}
    \item Zeitplan erstellt
    \item Analyse der Aufgabenstellung
    \item Kurzfassung und Verfeinerung des Auftrags
  \end{itemize}
  \\ \hline

  Aufgetretene Probleme      &
  Keine
  \\ \hline

  Durchgeführte Tests        &
  Noch keine
  \\ \hline

  Wissensbeschaffung         &
  \begin{description}
    \item[PkOrg] Aufgabenstellung und allgemeine Informationen zur IPA
    \item[Overleaf] Dokumentation zu Grundlegenden LaTeX Commands
  \end{description}
  \\ \hline

  Beanspruchte Hilfeleistung &
  Keine
  \\ \hline

  Nacht- und Wochenendarbeit &
  Auseinandersetzung mit Grafiken/Bilder in LaTeX
  \\ \hline

  Vergleich mit dem Zeitplan &
  Heute liege ich super im Zeitplan und zeitlich lief alles genau so wie geplant.
  Die Systemabgrenzung konnte ich noch nicht vollständig abschliessen, aber das schaffe ich morgen früh locker noch fertig.
  \\ \hline

  Reflexion                  &
  Der Einstieg in die IPA lief meiner Meinung nach relativ gut. Ich konnte meine Ziele heute erreichen, auch wenn ich gerne
  noch weiter gekommen wäre. Ich habe im IPA-Bericht teilweise ein wenig Zeit dadurch verloren, dass
  ich einige Male Sachen in der \emph{Overleaf} Dokumentation nachschlagen musste. Dennoch denke ich, dass es eine
  gute Entscheidung war, LaTeX zu verwenden.
  \\ \hline
\end{tabularx}
