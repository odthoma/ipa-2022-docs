\section{Tag 5 - 30.05.2022}

\begin{tabularx}{\textwidth}[H]{|c|X|}
    \hline
    Erledigte Arbeiten         &
    \textbf{Vormittag:}
    \begin{itemize}
        \item Views für das Einsehen und navigieren zwischen Tasks erstellt
        \item Optimierungen der Benutzereinladung, E-Mails erstellt
    \end{itemize}
    \textbf{Nachmittag:}
    \begin{itemize}
        \item Navigation zwischen den Aufgaben mittels \emph{Hotwire Turbo} implementieren
        \item Start mit der Implementation vom File-Upload
    \end{itemize}
    \\ \hline

    Erreichte Ziele            &
    \begin{itemize}
        \item Ansichten und Navigation zwischen den Tasks für Bewerber fertiggestellt 
    \end{itemize}
    \\ \hline

    Aufgetretene Probleme      &
    Es sind immer wieder kleinere Probleme aufgetreten in der Implementation von Turbo und ActiveStorage. Das war im allgemeinen etwas tricky,
    konnte aber mit der Hilfe der offiziellen Dokumentationen gut gelöst werden.
    \\ \hline

    Durchgeführte Tests        &
    Weiterhin wurde der neu geschriebene Code mit Unit- und Integrationtests getestet. Auch der neue Mailer für die Einladungen wurde getestet.
    \\ \hline

    Wissensbeschaffung         &
    \begin{description}
        \item[RailsGuides] Diverse Dokumentationen bzgl. ActiveRecord, ActionText, ActiveStorage und ActionMailer
        \item[RubyDoc] Regelmässiges nachschlagen von grundlegenden Funktionen
        \item[Hotwired] \enquote{Handbuch} von StimulusJS + TurboJS
        \item[GitHub - devise\_invitable] README von besagtem Gem
    \end{description}
    \\ \hline

    Beanspruchte Hilfeleistung &
    Klarstellung der Leitfrage 2.3 an Simon Huber. Es war nicht ganz klar, ob TurboDrive als ein \enquote{Site-Reload} zählen würde.
    \\ \hline

    Nacht- und Wochenendarbeit &
    Keine
    \\ \hline

    Vergleich mit dem Zeitplan &
    Meiner Meinung nach liege ich momentan super im Zeitplan und habe heute viel geschafft. Der kleine Rückstand ist in der Planung vorgesehen.
    \\ \hline

    Reflexion                  &
    Heute war ich sehr produktiv und konnte mich gut auf das Programmieren fokussieren. Das UI gefällt mir so weit sehr gut
    und ich denke, dass ich ein gutes Gesamtkonzept umgesetzt habe. Die Navigation zwischen den einzelnen Seiten scheint sehr intuitiv.
    Für die Einladungen der Benutzer habe ich etwas mehr Zeit gebraucht als erhofft, da ich nicht nur das \enquote{nackte} gem verwenden konnte.
    \\ \hline
\end{tabularx}
