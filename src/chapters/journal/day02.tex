\section{Tag 2 - 24.05.2022}

\begin{tabularx}{\textwidth}[H]{|c|X|}
    \hline
    Erledigte Arbeiten &
    \textbf{Vormittag:}
    \begin{itemize}
        \item Fertigstellen des Deployment-Diagramms für die Systemabgrenzung und Beschreibung der Schnittstellen/Systeme
        \item Erstellen und dokumentieren der Datenbankmodelle
        \item Beginn mit Zustandsdiagramm 
    \end{itemize}
    \textbf{Nachmittag:}
    \begin{itemize}
        \item Fertigstellung Zustandsdiagramm
        \item Mock-Ups Erstellen
    \end{itemize}
    \\ \hline

    Erreichte Ziele &
    \begin{itemize}
        \item Datenbankdiagramme erstellen
        \item Zustandsdiagramm eines Assessments Zeichnen
        \item Erste Entwürfe und Konzepte der Applikation
    \end{itemize}
    \\ \hline

    Aufgetretene Probleme &
    Keine
    \\ \hline

    Durchgeführte Tests &
    Noch keine
    \\ \hline

    Wissensbeschaffung &
    \begin{description}
        \item[Sparx und IBM] Informationen zu den UML 2.0 Standards
        \item[Overleaf] Dokumentation zu Tabellen, Farben und Referenzen
        \item[PkOrg] Aufgabenstellung und Anforderungen bzgl. IPA-Bericht
    \end{description}
    \\ \hline

    Beanspruchte Hilfeleistung &
    Kurzes Gespräch beim Kaffee mit Simon Huber über den Nutzen von UseCases
    \\ \hline

    Nacht- und Wochenendarbeit &
    Keine
    \\ \hline

    Vergleich mit dem Zeitplan &
    Grundsätzlich bin ich noch prima im Zeitplan, allerdings haben mich die Mock-Ups und die ganzen Diagramme teilweise
    sehr viel Zeit gekostet. Auch das Fertigstellen
    \\ \hline

    Reflexion &
    Im grossen und Ganzen bin ich zufrieden mit dem heutigen Tag. Allerdings hätte ich gerne die Planungsphase bereits abgeschlossen.
    Trotzdem bin ich froh, die ganze Zeit für die Planung aufgewendet zu haben. Ich denke,
    dass die Entwürfe und Diagramme mir die Realisierung erleichtern werden. Wenn alles nach Zeitplan läuft,
    kann ich morgen Nachmittag mit dem Programmieren starten.
    \\ \hline
\end{tabularx}
