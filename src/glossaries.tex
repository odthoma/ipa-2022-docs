% see B6.6
\newglossaryentry{Organigramm}
{
  name={Organigramm},
  description={Ein Organigramm stellt eine Organisation und deren Aufbauorganisation grafisch dar.}
}

\newglossaryentry{cicd}
{
  name={CI/CD},
  description={Bei der \emph{Continuous Integration} (CI) wird der Source Code mit dem restlichen Source Code zusammengeführt, gebaut und getestet.
      Beim \emph{Continuous Deployment} (CD) wird das Software Paket automatisch in die Produktion deployed.}
}

\newglossaryentry{dsl}
{
  name={DSL},
  description={Domain Specific Language ist eine Computersprache, die auf einen bestimmten Anwendungsbereich spezialisiert ist. Das bedeutet, dass sie entwickelt wurde, um spezifische Anforderungen zu erfüllen.}
}

\newglossaryentry{gem}
{
  name={Gem},
  description={Ein RubyGem ist eine Softwarebibliothek, die in eine Ruby-Applikation importiert werden kann. Diese enthält meistens eine Bestimmte Funktionalität sowie alle mit dieser Funktionalität verbundenen Dateien/Assets.}
}

\newglossaryentry{enum}
{
  name={Enum},
  description={Ein Enumerator ist ein Datentyp, der nur einen von verschiedenen vordefinierte Werten annehmen kann.}
}

\newglossaryentry{mvc}
{
  name={MVC},
  description={Das Akronym für Model View Controller, ein Softwarearchitektur-Konzept zur Aufteilung und Trennung von Komponenten.}
}

\newglossaryentry{crud}
{
  name={CRUD},
  description={Create, Read, Update und Delete. Dies sind die grundlegenden Aktionen, die in einer Datenbank durchgeführt werden können.}
}

\newglossaryentry{vcs}
{
  name={VCS},
  description={Ein Version Control System ist ein Wekzeug (z.B. Git), um Änderungen an Programmcode zu verfolgen und zu verwalten.}
}


\newglossaryentry{cors}
{
  name={CORS},
  description={Cross-Origin Resource Sharing beschreibt eine Funktion, mit der ein Webbrowser auf die Daten einer fremden Domain zugreifen kann.}
}

\newglossaryentry{happypath}
{
  name={Happy-Path},
  description={Beim Happy-Path Testing werden Testszenarien ausgewählt, die denjenigen entsprechen sollten, die ein Benutzer bei der normalen Nutzung der Applikation durchführen würde.}
}
