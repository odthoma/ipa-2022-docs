% see B6.6
\newglossaryentry{Organigramm}
{
  name={Organigramm},
  description={Ein Organigramm stellt eine Organisation und deren Aufbauorganisation grafisch dar.}
}

\newglossaryentry{cicd}
{
  name={CI/CD},
  description={Bei der \emph{Continuous Integration} (CI) wird der Source Code mit dem restlichen Source Code zusammengeführt, gebaut und getestet.
      Beim \emph{Continuous Deployment} (CD) wird das Software Paket automatisch in die Produktion deployed.}
}

\newglossaryentry{dsl}
{
  name={DSL},
  description={Domain Specific Language ist eine Computersprache, die auf einen bestimmten Anwendungsbereich spezialisiert ist. Das bedeutet, dass sie entwickelt wurde, um spezifische Anforderungen zu erfüllen.}
}

\newglossaryentry{gem}
{
  name={gem},
  description={Ein RubyGem ist eine Softwarebibliothek, die in eine Ruby-Applikation importiert werden kann. Diese enthält meistens eine Bestimmte Funktionalität sowie alle mit dieser Funktionalität verbundenen Dateien/Assets.}
}

\newglossaryentry{enum}
{
  name={Enum},
  description={Ein Enumerator ist ein Datentyp, der nur einen von verschiedenen vordefinierte Werten annehmen kann.}
}
